% #############################################################################
% This is Chapter 5
% !TEX root = ../main.tex
% #############################################################################
% Change the Name of the Chapter i the following line
\fancychapter{Pathology detection from speech}
\label{chap:final}
\cleardoublepage

As mentioned in the previous section, SLT can assist paediatric speech therapists by automatically assessing pronunciation quality and identifying pathological conditions. Although the primary aim of this thesis was to improve ASR for reliable assessment of pronunciation quality. We also contributed to the identification of pathological conditions from speech, which will be discussed in this section. 

The potential of speech as a non-invasive biomarker for evaluating a speaker's health for both physical and psychological disorders has repeatedly been proven by the results of several works  \cite{hauptman2019identifying,botelho2019speech}. Traditional speech-based disease classification systems have focused on carefully researched, knowledge-based features. However, these features do not always capture the full disease's symptomatology and may even ignore some of its more subtle signs. This has led research to move towards generic representations that intrinsically model the symptoms. However, there are not enough pathological speech data available to train a large model directly. In our work \cite{botelho2020pathological}, we proposed to assess speaker embedding, such as \textit{i-vectors} \cite{ivector} and \textit{x-vectors} \cite{snyder2018x}, applicability as a generic feature extraction method to the detection of Parkinson’s disease (PD) and Obstructive Sleep Apnea (OSA). All disease classifications were performed with a support-vector-machine (SVM) classifier. Our experiments with European Portuguese datasets support the hypothesis that discriminative speaker embeddings contain information relevant to disease detection. In particular, we found evidence that these embeddings contain information that hand-crafted features fail to represent, thus proving the validity of our approach. It was also observed that x-vectors are more suitable than i-vectors for tasks whose domain does not match the training data, such as verbal task mismatch and cross-lingual. This indicates that x-vectors embeddings are a strong contender in the replacement of knowledge-based feature sets for PD and OSA detection.

Later, in \cite{pompili2020inesc}, we proposed to extend the aforementioned work by classifying Alzheimer's disease with the conjunction of both acoustic and textual feature embeddings. In this end, speech signals are encoded into \textit{x-vector} using pre-trained models. For textual input, contextual embedding vectors are first extracted using an English Bert model \cite{Bert} and then used to feed a bidirectional recurrent neural network with attention. This multi-model system, based on the combination of linguistic and acoustic information, attained a classification accuracy of 81.25\%. Results have shown the importance of linguistic features in the classification of Alzheimer’s disease, which outperforms the acoustic ones in terms of accuracy.

Finally, we further extend the idea of using pre-trained representation to automatically detect COVID-19 from cough recordings. We leverage transfer learning to develop a set of COVID-19 classification subsystems based on deep cough representation extractors called experts. Individual decisions of three experts are fed to a calibrated decision-level fusion system. This ensemble of expert subsystems based on cough representations is expected to produce well-calibrated log-likelihood scores over a wide range of operating points. The output can be more easily interpreted by a human expert and incorporated into the decision-making process. Our results show competitive performance compared to hand-crafted features, although they are still far from those required to become a reliable tool to assist COVID-19 screening.
