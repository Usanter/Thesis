% #############################################################################
% Abstract Text
% !TEX root = ../main.tex
% #############################################################################
% reset acronyms
\acresetall
% use \noindent in firts paragraph
%\noindent Speech is a fundamental communication skill used for social interaction, express feelings and needs, among others. Unfortunately, there is an increasing number of people who suffer from debilitating speech pathologies, including children. A childhood speech disorder that is not properly diagnosed and treated can have long-term negative effects in social, communication and educational situations. Hence the importance of speech therapy.

%In recent years, machine learning has proven to have many applications in the health field, both for diagnosis and monitoring. Particularly, there is a growing interest in the development of automatic tools to assist speech-language pathologists. Furthermore, some automatic tools allow patients to do exercises outside the session with the therapist, for example at home. In the case of children, the exercises can be provided in a gamification context, which contributes to motivate them to practice more often. Most of these tools require reliable automatic speech recognition systems, as these are commonly used to provide pronunciation quality scores. However, despite recent significant improvements in the performance of automatic speech recognition systems for healthy adults, these systems' performance dramatically drops when considering speech from children and speakers with speech pathologies. This is mainly due to the high acoustic variability and the reduced amount of training data available.

%In this proposal, we propose to investigate how machine learning algorithms in speech recognition tools for children can enable more complex exercises and better feedback. We first present a summary of existing automatic speech recognition techniques for children. In view of the need for a robust oracle model for children, this proposal will mainly focus on improving the automatic speech recognition system for children. Therefore, we identified several methods for improvement. i) Knowledge transfer on Hybrid and end-to-end models by using transfer and multi-task learning and ii) Adapter transfer in end-to-end speech recognition for a parameter efficient transfer.
%: i) Transfer learning methods, using adult and children speech knowledge with weights transfer and multi-task learning. ii) Adapter transfer for light and efficient adaptation.
%Data augmentation based on speech synthesis of children speech.

%\noindent In recent years, Automatic Speech Recognition (ASR) technology has advanced significantly, opening avenues for novel applications targeting young speakers. Potential use cases include innovative man-machine interactions through voice, automatic reading tutors, and speech pathologist assistants among others. However, the unique challenges presented in children's speech, characterised by high variability in both acoustics and linguistics components, impede the recognition performance of existing adult ASR systems.

%This thesis focuses on improving ASR capabilities specifically tailored for children, aiming to develop a more robust and accurate system calledapable of handling the variabilities in speech across different age groups. To this end, various knowledge transfer approaches were explored.

%The investigation initiated with knowledge transfer on Hybrid speech recognition models, employing transfer and multi-task learning strategies, and their combination. Subsequently, an end-to-end granular fine-tuning method was introduced to nhance and understand the adaptability of ASR systems to the nuances of children's speech during transfer learning. In parallel, Adapter transfer was examined alongside other parameter-efficient transfer techniques in order to discover a parameter efficient transfer method. Additionally, a novel approach involving data augmentation through synthetic data was explored to further enhance generalisation to children's speech patterns.

%This research makes a significant contribution to the field of children's speech technology, providing a deeper understanding of knowledge transfer processes and introducing innovative approaches. The outcomes pave the way for improved human-computer interactions in educational, entertainment, and assistive technology applications specifically tailored for children. These results pave the way for future advances in ASR technology specially designed for the unique characteristics of young speakers.


\noindent This thesis explores strategies to overcome the inherent challenges associated with recognising children's speech, particularly focusing on the variability in acoustics and the limited availability of data. It begins with a comprehensive overview of existing automatic speech recognition (ASR) techniques tailored for children.

Subsequently, the thesis delves into the development of a hybrid ASR system using knowledge transfer approaches, specifically targeting European Portuguese and English. Transfer learning emerges as a particularly effective method in this context. Additionally, this thesis introduces the novel method of Multilingual transfer learning, combining multi-task learning with transfer learning, which proves to be superior in low-resource scenarios.

Transitioning towards the end-to-end paradigm, this thesis investigate the role of the Encoder in fine-tuning for children's ASR, underscoring the significance of addressing acoustic variability. A novel partial fine-tuning approach for Transformer-based architectures is proposed, demonstrating superior performance compared to traditional entire model fine-tuning.

Further investigation focuses the use of Adapter modules for parameter-efficient transfer learning, showcasing their effectiveness over full model fine-tuning for children's ASR. Additionally, unsupervised clustering of utterances is employed to enhance Adapter performance, revealing their potential for group-specific adaptation.

Building upon the efficiency of Adapters, the thesis introduces the Double Way Adapter Tuning method, leveraging Text-to-speech data for data augmentation. This technique significantly reduces the gap between synthetic and real speech during fine-tuning, resulting in notable enhancements in ASR performance.

Lastly, the thesis evaluates various parameter-efficient methods, ultimately proposing the concept of Shared-Adapters, where one Adapter is shared across all layers. Despite minor score degradation, Shared-Adapters offer superior parameter efficiency transfer compared to traditional methods, making them a compelling choice for children's ASR models.

In conclusion, this thesis offers comprehensive insights and innovative methodologies to tackle the challenges associated with children's ASR, thereby contributing significantly to the advancement of the field.
\newpage