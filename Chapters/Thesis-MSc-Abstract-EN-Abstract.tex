% #############################################################################
% Abstract Text
% !TEX root = ../main.tex
% #############################################################################
% reset acronyms
\acresetall
% use \noindent in firts paragraph
\noindent Speech is a fundamental communication skill used for social interaction, express feelings and needs, among others. Unfortunately, there is an increasing number of people who suffer from debilitating speech pathologies, including children. A childhood speech disorder that is not properly diagnosed and treated can have long-term negative effects in social, communication and educational situations. Hence the importance of speech therapy.

In recent years, machine learning has proven to have many applications in the health field, both for diagnosis and monitoring. Particularly, there is a growing interest in the development of automatic tools to assist speech-language pathologists. Furthermore, some automatic tools allow patients to do exercises outside the session with the therapist, for example at home. In the case of children, the exercises can be provided in a gamification context, which contributes to motivate them to practice more often. Most of these tools require reliable automatic speech recognition systems, as these are commonly used to provide pronunciation quality scores. However, despite recent significant improvements in the performance of automatic speech recognition systems for healthy adults, these systems' performance dramatically drops when considering speech from children and speakers with speech pathologies. This is mainly due to the high acoustic variability and the reduced amount of training data available.

In this proposal, we propose to investigate how machine learning algorithms in speech recognition tools for children can enable more complex exercises and better feedback. We first present a summary of existing automatic speech recognition techniques for children. In view of the need for a robust oracle model for children, this proposal will mainly focus on improving the automatic speech recognition system for children. Therefore, we identified several methods for improvement. i) Knowledge transfer on Hybrid and end-to-end models by using transfer and multi-task learning and ii) Adapter transfer in end-to-end speech recognition for a parameter efficient transfer.
%: i) Transfer learning methods, using adult and children speech knowledge with weights transfer and multi-task learning. ii) Adapter transfer for light and efficient adaptation.
%Data augmentation based on speech synthesis of children speech.
\newpage