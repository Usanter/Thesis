% #############################################################################
% This is Appendix A
% !TEX root = ../main.tex
% #############################################################################
\chapter{Transfer Learning-Based Cough Representations for Automatic Detection of COVID-19}
\label{chapter:appendixA}
Work done by: Rubén Solera-Ureña$^1$, Catarina Botelho$^{1,2}$, Francisco Teixeira$^{1,2}$, Thomas Rolland$^{1,2}$, \\Alberto Abad$^{1,2}$, Isabel Trancoso$^{1,2}$\\
 $^1$INESC-ID, Lisbon, Portugal\\
  $^2$Instituto Superior Técnico, University of Lisbon (IST-UL), Portugal

\section{Introduction}
In the last months, there has been an increasing interest in developing reliable, cost-effective, immediate and easy to use machine learning based tools that can help health care operators, institutions, companies, etc. to optimize their screening campaigns. In this line, several initiatives emerged aimed at the automatic detection of COVID-19 from speech, breathing and coughs, with inconclusive preliminary results. The ComParE 2021 COVID-19 Cough Sub-challenge provides researchers from all over the world a suitable test-bed for the evaluation and comparison of their work. In this paper, we present the INESC-ID contribution to the ComParE 2021 COVID-19 Cough Sub-challenge. We leverage transfer learning to develop a set of three expert classifiers based on deep cough representation extractors. 
A calibrated decision-level fusion system provides the final classification of coughs recordings as either COVID-19 positive or negative. Results show unweighted average recalls of 72.3\% and 69.3\% in the development and test sets, respectively. Overall, the experimental assessment shows the potential of this approach although much more research on extended respiratory sounds datasets is needed.
