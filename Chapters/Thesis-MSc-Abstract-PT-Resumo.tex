% #############################################################################
% RESUMO em Português
% !TEX root = ../main.tex
% #############################################################################
% use \noindent in firts paragraph
% reset acronyms
\acresetall
\noindent Nos últimos anos, a tecnologia de Reconhecimento Automático da Fala (RAF) tem avançado, abrindo portas para novas aplicações destinadas a jovens falantes. Potenciais casos de uso incluem interações inovadoras homem-máquina através da voz, tutores de leitura automática e assistentes de patologia da fala. Contudo, os desafios únicos apresentados pela fala das crianças, com uma elevada variabilidade nas componentes acústica e linguística, dificultam o desempenho dos atuais sistemas RAF para adultos.

Esta tese foca-se na melhoria das capacidades de RAF adaptadas às crianças, com o objetivo de desenvolver um sistema mais robusto e preciso capaz de lidar com as variabilidades da fala em diferentes grupos etários. Foram exploradas várias abordagens de transferência de conhecimentos.

A investigação começou com a transferência de conhecimentos sobre modelos híbridos de reconhecimento da fala, utilizando estratégias de transferência e de aprendizagem multitarefa, assim como a sua combinação. Posteriormente, introduziu-se um método de afinação granular de ponta a ponta para melhorar e compreender a adaptabilidade dos sistemas RAF às nuances do discurso das crianças durante a aprendizagem por transferência. Em paralelo, a transferência Adapter foi examinada juntamente com outras técnicas de transferência eficientes em termos de parâmetros, a fim de descobrir um método eficiente em termos de parâmetros. Adicionalmente, foi explorada uma nova abordagem que envolve o aumento de dados através de dados sintéticos para melhorar ainda mais a generalização aos padrões de fala das crianças.

Esta investigação contribui significativamente para o domínio da tecnologia de fala para crianças, proporcionando uma compreensão mais profunda dos processos de transferência de conhecimento e introduzindo abordagens inovadoras. Os resultados abrem caminho para melhores interações homem-computador em aplicações educativas, de entretenimento e de tecnologia assistiva especificamente concebidas para crianças, apontando para futuros avanços na tecnologia RAF desenvolvida para as características únicas dos jovens falantes.

\newpage